Podemos definir el ransomware como un tipo de malware que secuestra la máquina del usuario afectado con diversas técnicas pidiendo un rescate para poder recuperarla \cite{definicion-ransomware-2}. Los ransomware modernos y los más comunes suelen encriptar el ordenador o dispositivo consiguiendo con ello que los datos no estén disponibles para el propietario.\\
Para poder recuperar la máquina se utilizan diversos métodos pero por regla general se emplean métodos de pago que no puedan ser rastreados como los Bitcoins \cite{definicion-ransomware-3}. Gracias a esto los atacantes reciben dinero a través de operaciones monetarias a una cuenta anónima siendo las operaciones no rastreables.\\
Con esto se produce una extorsión hacia el usuario, ya que normalmente si no se paga antes de un periodo establecido los archivos son borrados. Esto produce pérdidas de información no sólo a particulares sino a grandes empresas y PYMES. Este tipo de ataque representa un problema cuando no se tienen mecanismos de copia de seguridad para prevenir pérdidas de información. Actualmente este es uno de los tipos de malware más común teniendo numerosos ejemplos como Reveton, CryptoLocker, CryptoWall o Wannacry \cite{definicion-ransomware-4}.