Dado que WannaCrypt es un ransomware, su funcionamiento está basado en la encriptación o cifrado del sistema de archivos del terminal infectado, o parte de él, de forma que el usuario es incapaz de acceder a su información (dando lugar a la extorsión, chantaje o rescate). Para conseguirlo, los desarrolladores del malware han aunado los dos tipos de cifrado que existen, simétrico y asimétrico, dando lugar a lo que se conoce como cifrado mixto, aprovechando los puntos fuertes de ambos cifrados y solapando sus debilidades. En el caso del cifrado simétrico, el proceso de cifrado es muy rápido pero la clave que desencripta supone un problema, ya que debe mantenerse a buen recaudo para que el malware surta efecto y no es fácil. Por el contrario, el cifrado asimétrico, con el paradigma clave pública-privada solventa los problemas con la clave, pero la encriptación es muy lenta (debido a los algoritmos matemáticos necesarios para generar claves tan beneficiosas). \\

El procedimiento seguido por WannaCrypt es el siguiente: el software, al ejecutarse, comienza a encriptar con cifrado simétrico los archivos de las carpetas (Temp, por ejemplo, no es cifrada). Las claves de ese cifrado simétrico son, a su vez, cifradas con cifrado asimétrico, de forma que se genera una clave pública que encripta y una privada que desencripta. Esa clave privada es enviada a un servidor, totalmente desconocido para el usuario, que guarda la clave. En la práctica, la hace desaparecer. Si el computador infectado no estuviera conectado a la red, la clave privada permanece en el mismo. \\

Hemos comprobado, con nuestros experimentos sobre máquinas virtuales, que no se encriptan todos los archivos ni todas las carpetas. Además, aquellos que sí son encriptados son movidos a otros directorios del sistema de archivos, generando así un caos sin precedentes. \\

En \cite{wc-encrip} se puede ver la explicación de estos procedimientos por parte de un profesor de la Universidad de Nottingham