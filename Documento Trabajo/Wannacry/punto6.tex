Las reacciones de Microsoft sobre WannaCrypt no se hicieron esperar. La compañía hizo una petición a todos los gobiernos del mundo para ver el ciberataque global como una 'llamada de atención', como así asegura la nota de prensa de la agencia EFE en \cite{vg}. En palabras de Brad Smith, presidente y asesor legal de Microsoft, 'Hemos visto aparecer en WikiLeaks vulnerabilidades almacenadas por la CIA, y ahora estas vulnerabilidad robada a la NSA ha afectado a clientes en todo el mundo'. De aquí podemos extraer que WikiLeaks filtró dicho fallo de seguridad y los atacantes la usaron para así confeccionar su malware.

En \cite{elhacker} se pueden encontrar las posibles cuentas de Bitcoins que los secuestradores establecieron para cobrar los rescates (\cite{b1},\cite{b2},\cite{b3},\cite{b4}). En ellas se muestra que han podido cobrar alrededor de 139,85063858 BTC, lo que hace un total de 280590,94\euro