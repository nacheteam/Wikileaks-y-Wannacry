En la programación del malware Wannacrypt0r se pensó en el momento en que se requiriera parar el ransomware completamente y con ello desactivarlo. Este procedimiento ha sido llamado por algunos expertos como 'KillSwitch'.\\
El malware, al ejecutarse en la máquina atacada, lo primero que realiza es comprobar un dominio que en un principio no se encontraba registrado. Este dominio puede ser extraído de diferentes maneras del binario. Para poder extraer la información en este caso podemos utilizar herramientas como 'string', 'pestr' o incluso Wireshark. Con las dos primeras herramientas podemos obtener cadenas de texto que se encuentran directamente en el fichero ejecutable. Al hacer esto podemos observar una única referencia web en texto plano que resultó ser el llamado 'KillSwitch'. El otro modo de obtener dicha dirección es con un sniffing de la red para ver con qué dominios y servidores se comunica el malware.\\
Este proceso fue realizado por un forense informático que también dirige un blog famoso sobre malware \cite{killswitch-situacion-2}. Tras darse cuenta de la existencia de este dominio introducido dentro del binario lo registró para ver si ocurría algo y con ello paró la extensión de Wannacry \cite{killswitch-situacion-1}.\\
Actualmente el malware continúa expandiéndose pero no con su versión original que ya ha sido desactivada. Se están realizando copias de dicho ransomware eliminando el KillSwitch y realizando variantes, incluso pidiendo más dinero \cite{killswitch-situacion-3}. Así mismo ya disponemos de información obtenida del malware \cite{killswitch-situacion-4} y de ciertas herramientas en desarrollo para revertir el proceso de encriptado como la desarrollada por Eleven Paths(Telefónica) \cite{killswitch-situacion-5} o por usuarios anónimos de GitHub \cite{killswitch-situacion-6}.