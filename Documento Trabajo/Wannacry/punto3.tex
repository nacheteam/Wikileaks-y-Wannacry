Este ataque ha sido programado y planificado para atacar máquinas ejecutando los sistemas operativos de Microsoft, tanto en sus versiones de escritorio como en sus versiones de servidor \cite{plataformas-contagio-2}. Entre las versiones afectadas encontramos: Windows Vista, Windows Server 2008, Windows 7, Windows Server 2008 R2, Windows 8.1, Windows Server 2012, Windows 10, Windows Server 2012 R2, Windows Server 2016, Windows XP, Windows 8.\\
La versión más problemática de la lista es Windows 7 que aún siendo una versión ya antigua de Windows tiene actualmente el 48.5\% de cuota de mercado \cite{plataformas-contagio-3}, lo que hace que la mayoría de las personas que han sido infectadas con Wannacry estuvieran ejecutando Windows 7.\\
Al igual que la mayoría de ransomware se instala en la máquina afectada a través de correos electrónicos con ficheros maliciosos. Estos ficheros pueden ser hojas de cálculo con programas embebidos, archivos PDF o directamente ficheros ejecutables. En este proceso se aprovechan vulnerabilidades o puertas traseras que permitan a los atacantes introducir su malware en los formatos mencionados anteriormente \cite{plataformas-contagio-1}.\\
Tras esta infección inicial los atacantes utilizan una vulnerabilidad conocida del protocolo SMB para continuar la infección dentro de la red local \cite{plataformas-contagio-4}. Este protocolo se utiliza para compartir ficheros entre máquinas Windows dentro de la misma red \cite{plataformas-contagio-5}. La vulnerabilidad permite ejecutar código remoto en otra máquina de la misma red pudiendo de esta manera los atacantes ejecutar el binario de Wannacrypt0r en el resto de máquinas de la red para infectarlos a todos.