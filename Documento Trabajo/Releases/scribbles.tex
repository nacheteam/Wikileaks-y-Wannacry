El 28 de Abril de 2017 tuvo lugar la publicación de la release 'Scribbles' \cite{scribbles}. Estos documentos liberados por WikiLeaks hacen referencia a un proyecto de la CIA sobre marcas de agua en documentos. Este procedimiento indica cómo insertar 'Web beacon tags' dentro de un documento. Estas etiquetas son consideradas por la EFF como un bug que provoca un fallo de seguridad hacia el usuario. Estas etiquetas pueden camuflarse, incluso aparecer invisibles dentro de los documentos por lo que pasarían inadvertidas ante los usuarios. Las etiquetas tienen la función de monitorizar quién está leyendo un documento \cite{web-beacon-tag}. \\
Según WikiLeaks y la documentación del proyecto Scribbles ha sido útil sobre todo en las versiones de Microsoft Office 97-2016. Este fallo de seguridad afectaría a los documentos creados con esta suite ofimática, por contra no se garantiza el funcionamiento bajo documentos OpenOffice o LibreOffice ya que estos procesadores de texto podrían revelas las etiquetas al usuario y así poder evitarlas.\\
Cabe además destacar que este procedimiento se conoce desde el 1 de marzo de 2016 y no se pretendía liberar hasta (como mínimo) 2066.