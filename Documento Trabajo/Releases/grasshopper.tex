El 7 de abril de 2017, WikiLeaks publicó la release Grasshopper \cite{grasshopper}, una plataforma usada para confeccionar la parte maliciosa del malware para sistemas operativos Windows. 

Grasshopper se nutre de múltiples módulos que pueden ser usados por los operadores de la CIA como bloques para construir un implante personalizado que se comportará de forma diferente, por ejemplo manteniendo persistencia en un computador, dependiendo de qué particularidades o capacidades sean seleccionadas en el proceso de construcción. Adicionalmente, Grasshopper ofrece un lenguaje muy flexible para definir reglas usadas para supervisar el dispositivo objetivo, de forma que el payload solo se instala si el objetivo tiene la configuración correcta, por ejemplo, si el computador ejecuta una versión de Windows específica, o si tiene un antivirus en concreto.

Grasshopper ofrece herramientas usando ciertos mecanismos de persistencia y modificados a través de extensiones. Uno de los mecanismos de persistencia utilizados por la CIA es 'Stolen Goods', cuyos componentes fueron tomados del malware Carberp,  un rootkit ruso, confirmando el supuesto reciclado de malware encontrado por la CIA. 'El código fuente de Carberp fue publicado en Internet y no ofrece dificultades para robar cualquiera de sus componentes'.