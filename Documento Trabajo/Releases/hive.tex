HIVE \cite{hive} fue publicada el 14 de abril de 2017. Se trata de toda una infraestructura malware back-end con una interfaz de acceso público HTTPS usada por la CIA para transferir información de ciertas máquinas a la CIA y recibir comandos de sus operadores para ejecutar tareas específicas en dichas máquinas. HIVE es utilizado a través de múltiples softwares maliciosos y operaciones de la CIA. La interfaz pública HTTPS utiliza zonas poco sospechosas para así ocultar su presencia. Tiene dos funciones principales: monitorización de los usuarios y shell interactiva. \\

Se puede encontrar una descripción de esta infraestructura back-end en \cite{symantec}, donde se dice que para servidores C\&C (Command and Control), se configuran dominios específicos y direcciones IP para cada objetivo. Estos dominios parece que son registrados por los atacantes, pero en realidad se usan servicios de privacidad para ocultar su identidad real. Las IPs provienen de compañías que ofertan VPS o servicios de webhosting. El malware se comunica con los servidores C\&C a través de HTTPS usando un protocolo de criptografía para impedir la identificación de las comunicaciones. 

