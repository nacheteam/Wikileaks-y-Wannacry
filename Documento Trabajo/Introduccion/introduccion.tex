Hasta ahora, toda nuestra interacción con los computadores y programas ha venido determinada únicamente por el mero hecho de la funcionalidad. Hemos basado nuestros sistemas de comunicación como Internet únicamente en el aspecto funcional, sin tener la precaución suficiente en temas de privacidad, vulnerabilidades y seguridad informática en general. \\
Este punto puede provocar que nuestros servicios web, servicios de almacenamiento de ficheros o aplicaciones profesionales fallen, liberen información confidencial y con ello expongan a nuestra empresa. \\
WikiLeaks ha puesto a disposición de todos una serie de vulnerabilidades y herramientas que han sido supuestamente empleadas por agencias de inteligencia como la CIA, el FBI y la NSA. Entre estas vulnerabilidades ha destacado recientemente un fallo de seguridad del protocolo SMB que ha infectado numerosos equipos con un ransomware llamado Wannacrypt0r. Este fallo ha tenido a muchas empresas en una situación difícil, entre ellas podemos mencionar a Telefónica, Renault o a Iberdrola.