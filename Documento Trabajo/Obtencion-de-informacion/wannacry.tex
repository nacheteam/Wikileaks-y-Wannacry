En el caso de Wannacry, al estar muy reciente el hecho, hemos tenido que hacer una investigación del malware y su funcionamiento por nuestra cuenta. Hemos querido investigar varios factores como el funcionamiento, cómo podemos desactivarlo con las herramientas que existen actualmente en desarrollo y qué información podemos obtener en sí del fichero ejecutable.\\
En primer lugar tuvimos que obtener una muestra del malware fiable. Nuestra muestra fue obtenida de un Gist en GitHub que contenía información del malware \cite{killswitch-situacion-4}. Comprobamos mediante una plataforma fiable como Virus Total que el código MD5 de la muestra que habíamos obtenido estaba catalogado como Wannacry en su base de datos.\\
Tras esto ya podemos empezar a trabajar con el malware. Para cerciorarnos de que nuestras máquinas no sufrían daños ni se veían afectadas por el malware decidimos probar el ransomware en una máquina virtual con Windows 7, ya que esta ha sido la plataforma más afectada y se especula que se desarrolló pensando en la misma. Por motivos de seguridad deshabilitamos cualquier adaptador de red de la máquina virtual ya que esto podía hacer que infectásemos equipos Windows dentro de la misma red que aún no tuvieran el parche. Al tener los adaptadores de red desactivados también evitamos que el malware se desactive al comprobar su KillSwitch (Ver ~\ref{KillSwitch}). Así mismo por precaución debemos eliminar cualquier carpeta compartida que hayamos creado, sacar cualquier disco virtual que hayamos introducido y deshabilitar el portapapeles compartido y la función de arrastrar y soltar. Gracias a toda la configuración podemos estar seguros de que contendremos el malware dentro de la máquina virtual y no infectará ni encriptará nuestro sistema. Somos conscientes de que utilizar una máquina virtual podría haber llevado a problemas ya que no funciona como un 'sandbox', de todos modos y tras realizar varias pruebas hemos concluido que el malware no explota ninguna vulnerabilidad que le permita infectar la máquina anfitriona desde la máquina virtual.\\
Una vez establecidos todos los patrones de seguridad hemos querido obtener las cadenas embebidas dentro del fichero ejecutable. Para ello hemos creado por seguridad una nueva máquina virtual con una distribución GNU/Linux con configuración idéntica a la máquina anterior. De este modo nos ahorramos trabajar con el malware en nuestra máquina e infectarla por error. Dentro de esta máquina virtual hemos ejecutado el comando 'strings' sobre nuestra muestra. De aquí hemos obtenido información como el KillSwitch, las plataformas sobre las que funciona y varios nombres de librerías que emplea.\\
Para deducir el funcionamiento de Wannacry lo hemos probado en la máquina con Windows 7 que describimos anteriormente. Gracias a esto hemos podido observar 'in situ' los ficheros que se crean, el mensaje de alerta del ransomware y la manera en que encripta los archivos (Ver ~\ref{Funcionamiento-Wannacry})\\
Para desactivarlo hemos probado varias herramientas citadas en el apartado ~\ref{KillSwitch}. Hemos podido concluir que las herramientas que actualmente están desarrolladas no funcionan de manera estable y los ficheros pueden no ser recuperados. Entre las herramientas probadas hemos podido ver que la que mejor resultado nos otorga es un script realizado en PowerShell por la empresa Eleven Paths \cite{killswitch-situacion-5}.