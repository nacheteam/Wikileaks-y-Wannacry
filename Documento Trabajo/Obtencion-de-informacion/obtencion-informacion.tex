Actualmente podemos obtener mucha información acerca de Vault 7 y las releases que se han hecho hasta la fecha. Cabe destacar que conforme van pasando los días se van publicando nuevas releases y archivos con vulnerabilidades.\\
Para empezar podemos leer la nota de prensa que publicó WikiLeaks con motivo de la revelación de Vault 7 \cite{wikileaks-press-release}. En esta nota de prensa se explica con carácter general el motivo de la liberación del contenido de Vault 7 y qué contiene. Así mismo podemos encontrar en esa página un breve análisis, ejemplos y un apartado de preguntas frecuentes. Este apartado sirve como introducción al repositorio de documentos.\\
También disponemos de las subsecciones que hemos resumido con anterioridad. Cada subsección tiene su propia página que nos resume qué podemos encontrar y qué vulnerabilidad o ataque liberan en esa release. También disponemos, a parte de los resúmenes, de los documentos que conforman, explican y escenifican la vulnerabilidad o ataque que protagoniza la release \cite{releases-documents}. Dentro de estos documentos podemos encontrar guías de usuario, explicaciones extensas del fallo e incluso ejemplos de los ataques.\\
Estas secciones explicadas anteriormente conforman las vulnerabilidades más grandes y representativas pero disponemos aún de un repositorio más grande que contiene el resto de información de Vault 7. Este repositorio está formado por los ataques y fallos descubiertos por la comunidad y que han sido dados a WikiLeaks. Estos fallos pueden contener más o menos información pero al menos se nos informa de en qué consiste dicha vulnerabilidad. Además contiene más información sobre las releases principales de Vault 7 \cite{vault7-documents}.\\
Cabe mencionar al menos que para subir un fichero a WikiLeaks necesitaremos Tor. En su página web disponemos de una opción llamada 'Submit' que nos redirigirá a una página web que nos indica que debemos usar Tor y nos proporciona un enlace Onion \cite{vault7-submit}. Si abrimos este enlace en el Tor-Browser nos dará un formulario que podemos rellenar indicando el material que tenemos así como diferentes cuestiones que permiten identificar la información y comprobar su veracidad \cite{wikileaks-tor-submit}.