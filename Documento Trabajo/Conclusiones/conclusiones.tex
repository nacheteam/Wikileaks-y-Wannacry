Como hemos podido comprobar a lo largo de este trabajo la seguridad informática afecta tanto a usuarios como a servidores y empresas. Este hecho hace que debamos establecer reglamentos internos de seguridad, políticas de seguridad y prevenir las infecciones en nuestros entornos de trabajo y personal.\\
En el caso de los servidores y las empresas debemos de mantener copias de seguridad con periodicidad que nos permitan recuperar el flujo de trabajo y la información de manera rápida ante un ataque o un fallo. Asimismo debemos formar a nuestro personal de cualquier ámbito para saber qué archivos se pueden abrir con un ordenador de la empresa y cómo deben operar con los archivos que quieran abrir. Esto puede incluir usar por ejemplo versiones de programas para leer documentos que sepamos que son menos propensas a fallos de seguridad.\\
Estas políticas de seguridad podrían haber prevenido a muchas empresas de perder su información por culpa del ransomware tratado a lo largo de este trabajo. En el caso de haber sido infectado hubiera sido tan sencillo recuperarse del ataque como cargar la copia de seguridad disponible y continuar con el ritmo de trabajo. Esto puede hacernos perder alguna información, pero siempre es preferible esto antes que perder todo lo recopilado hasta el momento. \\
De igual manera los protocolos de seguridad podrían haber prevenido a muchas empresas de infectarse con este malware. Los empleados deben estar formados y comprender los riesgos que puede conllevar abrir un email de una persona desconocida en nuestro puesto de trabajo o ejecutar y abrir ficheros que no hemos comprobado. Una política de seguridad clara es que no se deben ejecutar ficheros que se hayan recibido a través de correo electrónico de usuarios no confiables.\\
Por lo tanto podemos concluir que la seguridad informática es un campo de la empresa y el entorno laboral que no debemos dejar de lado. Las prevenciones en estos entornos son clave y van a conducir a la larga a un entorno de trabajo más seguro y estable.